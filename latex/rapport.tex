\documentclass[]{report}

\usepackage[utf8]{inputenc}
\usepackage[T1]{fontenc}
\usepackage[francais]{babel}
\usepackage{graphicx}

% Title Page

	\title{Rapport de projet - Planificateur de réunions}
	
	\author{BELLANGER Stephen
		MONNIER Ysée
		VIALLA Maxence}
	\date{Septembre-Novembre 2016}



\begin{document}
\begin{titlepage}
	
	\includegraphics[scale=0.5]{figures/Logo_B00DLE.png}
	\maketitle
\end{titlepage}
\part{Rapport client}

\tableofcontents

\chapter{Introduction}

\paragraph{}
	L'objectif de ce projet est de réaliser un logiciel de planification de réunions en Java en se focalisant sur sa conception UML. L'interface et le mode de fonctionnement du logiciel sera basé sur le site web Doodle (http://www.doodle.com/). 
	\paragraph{}
	Le logiciel créé devra répondre au besoin de planifier des réunions en permettant à des utilisateurs de fournir une réponse sans devoir s'inscrire ni se connecter. Le logiciel devra fournir au créateur du sondage un accès facile aux réponses de tous les utilisateurs.
	\paragraph{}
	Ce rapport présentera la conception et la réalisation du logiciel B00DLE en respectant la chronologie de notre étude : spécifications, étude UML puis implémentation.



\chapter{Pré-conception}

\section{Dictionnaires}
\subsection{Termes}
\paragraph{Sondage} Un sondage est un tableau regroupant différent composant date/horaire ainsi que les réponses des différents utilisateurs.
\paragraph{Lien} Un lien est un identifiant unique pour retrouver un sondage.
\paragraph{Administrateur} Un administrateur est un utilisateur avec des droits supplémentaires. Il peut accéder à la création d’un sondage, à la modification et suppression d’un sondage qu’il a créé.
\paragraph{Utilisateur} Un utilisateur est une personne qui utilise l’application B00DLE sans compte administrateur.
\paragraph{Date} Une date correspond à un ensemble date en aaaa-mm-jj et une horaire en hh:mm.

\subsection{Actions}
\paragraph{Diffuser} Envoyer un e-mail du lien du sondage à l’ensemble des différentes adresses.
\paragraph{Clôturer} Terminer le droit de répondre à un sondage.
\paragraph{Sélectionner} Cocher dans une case à cocher une date.
\paragraph{Générer} Créer un lien unique qui correspond à un sondage.

\section{Spécifications}

\subsection{Résumé}
\subsubsection{Fonctionnelles}

\paragraph{Utilisateur}
\subparagraph{SPEC\_FONC\_002} Un utilisateur peut créer un compte
\subparagraph{SPEC\_FONC\_003} Un utilisateur peut se connecter à un compte existant
\subparagraph{SPEC\_FONC\_011} Un utilisateur peut modifier ses réponses
\subparagraph{SPEC\_FONC\_012} Un utilisateur peut supprimer ses réponses

\paragraph{Administrateur}
\subparagraph{SPEC\_FONC\_004} L’administrateur peut créer un sondage
\subparagraph{SPEC\_FONC\_005} L’administrateur peut gérer un sondage
\subparagraph{SPEC\_FONC\_006} L’administrateur peut clôturer un sondage
\subparagraph{SPEC\_FONC\_007} Sélectionner une date qui convient à l’ensemble des utilisateurs
\subparagraph{SPEC\_FONC\_014} Modifier les paramètres d’un sondage

\paragraph{Sondage}
\subparagraph{SPEC\_FONC\_008} Diffuser un sondage par mail
\subparagraph{SPEC\_FONC\_009} Générer le lien d’un sondage
\subparagraph{SPEC\_FONC\_010} Un utilisateur peut ajouter une réponse au sondage


\paragraph{Fonctionnalités spécifiques}
\subparagraph{SPEC\_FONC\_001} Système de chat
\subparagraph{SPEC\_FONC\_013} Prendre en charge un agenda personnel

\paragraph{Contraintes}
\subparagraph{CONT\_FONC\_001} Contrainte d’ajout d’une réponse à un sondage clôturé
\subparagraph{CONT\_FONC\_002} Contrainte de modification d’un sondage

\subsubsection{IHM}
\subparagraph{SPEC\_IHM\_001} Résumé textuel du sondage
\subparagraph{SPEC\_IHM\_002} Consultation d’un sondage (utilisateurs)
\subparagraph{SPEC\_IHM\_003} Consultation d’un sondage (administrateur)

\subsubsection{Transactionnelles}
\subparagraph{SPEC\_TRANS\_001} Envoyer le lien d’un sondage
\subparagraph{CONT\_TRANS\_001} Mise à jour du chat en temps réel

\subsubsection{Sécurité}
\subparagraph{SPEC\_SEC\_001} Le système requiert une authentification pour accéder aux droits administrateurs

\subsection{Spécifications fonctionnelles}

\subsubsection{Utilisateur}
\paragraph{SPEC\_FONC\_002 : Un utilisateur peut créer un compte.}
L’utilisateur doit entrer les informations suivantes pour créer un compte administrateur :
\begin{itemize}
\item Son nom
\item Son e-mail suivant le format d’un e-mail
\item Son mot de passe, au minimum 8 caractères
\item Vérification du mot de passe \end{itemize}
Il est automatiquement redirigé vers la page d’accueil du logiciel.

\paragraph{SPEC\_FONC\_003 : Un utilisateur peut se connecter à un compte existant.}
Pour qu’une authentification soit valide, l’utilisateur doit entrer :
\begin{itemize}
\item Son adresse mail
\item Son mot de passe \end{itemize}
L’adresse mail et le mot de passe doivent correspondre au données contenu dans la base de données.

\paragraph{SPEC\_FONC\_011 : Un utilisateur peut modifier ses réponses.}
Un utilisateur peut modifier les réponses à un sondage ayant été effectuées depuis le logiciel utilisé pour la création des réponses.

\paragraph{SPEC\_FONC\_012 : Un utilisateur peut supprimer ses réponses.}
Un utilisateur peut supprimer les réponses à un sondage ayant été effectuées depuis le logiciel utilisé pour la création des réponses.

\subsubsection{Administrateur}

\paragraph{SPEC\_FONC\_004 : Un administrateur peut créer un sondage.}
Un utilisateur peut créer un sondage en entrant les informations suivantes :
\begin{itemize}
\item Le nom du sondage 
\item La description du sondage
\item Le lieu du sondage
\item Ajouter les dates de début et de fin de chaque créneau  \end{itemize}

\paragraph{SPEC\_FONC\_005 : Un administrateur peut gérer un sondage.}
L’administrateur peut modifier les informations suivantes :
\begin{itemize}
\item Le nom du sondage
\item La description du sondage
\item Le lieu du sondage
\item Inviter une nouvelle personne au sondage \end{itemize}

\paragraph{SPEC\_FONC\_006 : Un administrateur peut clôturer un sondage. }
L’administrateur peut choisir de clôturer un sondage dont il est le créateur.

\paragraph{SPEC\_FONC\_007 : Un administrateur peut sélectionner une date qui convient à l’ensemble des utilisateurs.}
L’administrateur peut sélectionner une date valide qui lui convient par rapport aux réponses des utilisateurs. Il peut également clôturer un sondage sans faire de choix.

\paragraph{SPEC\_FONC\_014 : Un administrateur peut modifier les paramètres d’un sondage.}
Le système doit permettre à un administrateur authentifié de modifier tous les paramètres existant pour un sondage dont il est le créateur : \begin{itemize}
\item Modifier le nom, le lieu ou la description du sondage
\item Ajouter ou supprimer un créneau horaire
\item Suppression d’une ou plusieurs réponses d’utilisateur(s) \end{itemize}

\subsubsection{Sondage}

\paragraph{SPEC\_FONC\_008 : Le système peut diffuser un sondage.}
L’administrateur fournit au système une liste d’adresses mail via un formulaire. 

\paragraph{SPEC\_FONC\_009 : Le système génère le lien d’un sondage.}
Le système propose à l’administrateur un identifiant unique au sondage qu’il peut partager lui même avec les personnes de son choix.

\paragraph{SPEC\_FONC\_010 : Un utilisateur peut ajouter une réponse au sondage.}
Un utilisateur disposant du lien vers un sondage peut y ajouter une réponse. 

\subsubsection{Fonctionnalités spécifiques}

\paragraph{SPEC\_FONC\_001 : Le système intègre un chat instantané.}
Le système doit mettre en place un chat permettant aux utilisateurs de communiquer entre eux sans devoir s’authentifier. Chaque sondage dispose de son propre chat.

\paragraph{SPEC\_FONC\_013 : Le système peut prendre en charge un agenda personnel}
Le système doit pouvoir fournir à un utilisateur des indications lui rappelant qu’il a déjà répondu à un sondage. Cette indication bloque les dates indisponibles à tous les sondages auquel il est en train de répondre.

\subsubsection{Contraintes}
\paragraph{CONT\_FONC\_001 : Contrainte d’ajout d’une réponse à un sondage clôturé.}
Dans les 2 cas, il est impossible pour les utilisateurs d’ajouter une réponse à un sondage clôturé.

\paragraph{CONT\_FONC\_002 : Contrainte de modification d’un sondage.}
L’administrateur ne peut pas modifier une date de son sondage, ni modifier les réponses des utilisateurs.

\subsection{IHM}

\paragraph{SPEC\_IHM\_001 : Résumé textuel du sondage.}
Le système doit fournir un résumé textuel qui résume les disponibilités des utilisateurs.

\paragraph{SPEC\_IHM\_002 : Consultation d’un sondage (utilisateurs).}
Le système permet aux utilisateurs de consulter toutes les réponses à un sondage dont ils ont l’identifiant.

\paragraph{SPEC\_IHM\_003 : Consultation d’un sondage (administrateur).}
Le système permet à l’administrateur de consulter toutes les réponses à un sondage dont il est le créateur.

\subsection{Transactionnelles}

\paragraph{SPEC\_TRANS\_001 : Le système permet d'envoyer le lien d’un sondage.}
Le système envoie un lien aux utilisateurs spécifiés par l'administrateur permettant seulement de répondre au sondage.

\paragraph{CONT\_TRANS\_001 : Le système met à jour les chats en temps réel.}
Le chat est mis à jour en temps réel.

\subsection{Sécurité}
\paragraph{SPEC\_SEC\_001 : Le système requiert une authentification pour accéder aux droits administrateurs.}
Un utilisateur doit se connecter à un compte existant pour accéder à la création, modification ou suppression de sondage

\section{Maquettage}

\subsection{Sous-section}

\chapter{Conception UML}

\chapter{Implémentation}

\part{Diagrammes UML}

\part{Modélisation serveur}

\end{document}          
