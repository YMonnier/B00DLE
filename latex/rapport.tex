\documentclass[]{report}

\usepackage[utf8]{inputenc}
\usepackage[T1]{fontenc}
\usepackage[francais]{babel}
\usepackage{graphicx}

% Title Page

	\title{Rapport de projet - Planificateur de réunions}
	
	\author{BELLANGER Stephen
		MONNIER Ysée
		VIALLA Maxence}
	\date{Septembre-Novembre 2016}



\begin{document}
\begin{titlepage}
	
	\includegraphics[scale=0.5]{figures/Logo_B00DLE.png}
	\maketitle
\end{titlepage}
\part{Rapport client}

\tableofcontents

\chapter{Introduction}

\paragraph{}
	L'objectif de ce projet est de réaliser un logiciel de planification de réunions en Java en se focalisant sur sa conception UML. L'interface et le mode de fonctionnement du logiciel sera basé sur le site web Doodle (http://www.doodle.com/). 
	\paragraph{}
	Le logiciel créé devra répondre au besoin de planifier des réunions en permettant à des utilisateurs de fournir une réponse sans devoir s'inscrire ni se connecter. Le logiciel devra fournir au créateur du sondage un accès facile aux réponses de tous les utilisateurs.
	\paragraph{}
	Ce rapport présentera la conception et la réalisation du logiciel B00DLE en respectant la chronologie de notre étude : spécifications, étude UML puis implémentation.



\chapter{Pré-conception}

\section{Dictionnaires}
\subsection{Termes}
\paragraph{Sondage} Un sondage est un tableau regroupant différent composant date/horaire ainsi que les réponses des différents utilisateurs.
\paragraph{Lien} Un lien est un identifiant unique pour retrouver un sondage.
\paragraph{Administrateur} Un administrateur est un utilisateur avec des droits supplémentaires. Il peut accéder à la création d’un sondage, à la modification et suppression d’un sondage qu’il a créé.
\paragraph{Utilisateur} Un utilisateur est une personne qui utilise l’application B00DLE sans compte administrateur.
\paragraph{Date} Une date correspond à un ensemble date en aaaa-mm-jj et une horaire en hh:mm.

\subsection{Actions}
\paragraph{Diffuser} Envoyer un e-mail du lien du sondage à l’ensemble des différentes adresses.
\paragraph{Clôturer} Terminer le droit de répondre à un sondage.
\paragraph{Sélectionner} Cocher dans une case à cocher une date.
\paragraph{Générer} Créer un lien unique qui correspond à un sondage.

\section{Spécifications}

\subsection{Résumé}
\subsubsection{Fonctionnelles}

\paragraph{Utilisateur}
\subparagraph{SPEC\_FONC\_002} Un utilisateur peut créer un compte
\subparagraph{SPEC\_FONC\_003} Un utilisateur peut se connecter à un compte existant
\subparagraph{SPEC\_FONC\_011} Un utilisateur peut modifier ses réponses
\subparagraph{SPEC\_FONC\_012} Un utilisateur peut supprimer ses réponses

\paragraph{Administrateur}
\subparagraph{SPEC\_FONC\_004} L’administrateur peut créer un sondage
\subparagraph{SPEC\_FONC\_005} L’administrateur peut gérer un sondage
\subparagraph{SPEC\_FONC\_006} L’administrateur peut clôturer un sondage
\subparagraph{SPEC\_FONC\_007} Sélectionner une date qui convient à l’ensemble des utilisateurs
\subparagraph{SPEC\_FONC\_014} Modifier les paramètres d’un sondage

\paragraph{Sondage}
\subparagraph{SPEC\_FONC\_008} Diffuser un sondage par mail
\subparagraph{SPEC\_FONC\_009} Générer le lien d’un sondage
\subparagraph{SPEC\_FONC\_010} Un utilisateur peut ajouter une réponse au sondage


\paragraph{Fonctionnalités spécifiques}
\subparagraph{SPEC\_FONC\_001} Système de chat
\subparagraph{SPEC\_FONC\_013} Prendre en charge un agenda personnel

\paragraph{Contraintes}
\subparagraph{CONT\_FONC\_001} Contrainte d’ajout d’une réponse à un sondage clôturé
\subparagraph{CONT\_FONC\_002} Contrainte de modification d’un sondage

\subsubsection{IHM}
\subparagraph{SPEC\_IHM\_001} Résumé textuel du sondage
\subparagraph{SPEC\_IHM\_002} Consultation d’un sondage (utilisateurs)
\subparagraph{SPEC\_IHM\_003} Consultation d’un sondage (administrateur)

\subsubsection{Transactionnelles}
\subparagraph{SPEC\_TRANS\_001} Envoyer le lien d’un sondage
\subparagraph{CONT\_TRANS\_001} Mise à jour du chat en temps réel

\subsubsection{Sécurité}
\subparagraph{SPEC\_SEC\_001} Le système requiert une authentification pour accéder aux droits administrateurs
\subparagraph{SPEC\_SEC\_002} Cryptage du mot de passe administrateur. 

\subsection{Spécifications fonctionnelles}

\subsubsection{Utilisateur}
\paragraph{SPEC\_FONC\_002} Un utilisateur peut créer un compte.
L’utilisateur doit entrer les informations suivantes pour créer un compte administrateur :
\begin{itemize}
\item Son nom
\item Son e-mail suivant le format d’un e-mail
\item Son mot de passe, au minimum 8 caractères
\item Vérification du mot de passe \end{itemize}
Il est automatiquement redirigé vers la page d’accueil du logiciel.

\paragraph{SPEC\_FONC\_003} Un utilisateur peut se connecter à un compte existant.
Pour qu’une authentification soit valide, l’utilisateur doit entrer :
\begin{itemize}
\item Son adresse mail
\item Son mot de passe \end{itemize}
L’adresse mail et le mot de passe doivent correspondre au données contenu dans la base de données.

\paragraph{SPEC\_FONC\_011} Un utilisateur peut modifier ses réponses.
Un utilisateur peut modifier les réponses à un sondage ayant été effectuées depuis le logiciel utilisé pour la création des réponses.

\paragraph{SPEC\_FONC\_012} Un utilisateur peut supprimer ses réponses.
Un utilisateur peut supprimer les réponses à un sondage ayant été effectuées depuis le logiciel utilisé pour la création des réponses.

\subsubsection{Administrateur}

\section{Maquettage}

\subsection{Sous-section}

\chapter{Conception UML}

\chapter{Implémentation}

\part{Diagrammes UML}

\part{Modélisation serveur}

\end{document}          
