\documentclass[]{report}

\usepackage[utf8]{inputenc}
\usepackage[T1]{fontenc}
\usepackage[francais]{babel}
\usepackage{graphicx}

% Title Page

	\title{Rapport de projet - Planificateur de réunions}
	
	\author{BELLANGER Stephen
		MONNIER Ysée
		VIALLA Maxence}
	\date{Septembre-Novembre 2016}



\begin{document}
\begin{titlepage}
	
	\includegraphics[scale=0.5]{figures/Logo_B00DLE.png}
	\maketitle
\end{titlepage}
\part{Introduction}

\paragraph{}
	L'objectif de ce projet est de réaliser un logiciel de planification de réunions en Java en se focalisant sur sa conception UML. L'interface et le mode de fonctionnement du logiciel sera basé sur le site web Doodle (http://www.doodle.com/). 
	\paragraph{}
	Le logiciel créé devra répondre au besoin de planifier des réunions en permettant à des utilisateurs de fournir une réponse sans devoir s'inscrire ni se connecter. Le logiciel devra fournir au créateur du sondage un accès facile aux réponses de tous les utilisateurs.
	\paragraph{}
	Ce rapport présentera la conception et la réalisation du logiciel B00DLE en respectant la chronologie de notre étude : spécifications, étude UML puis implémentation.



\part{Pré-conception}

\chapter{Dictionnaires}
\chapter{Spécifications}
\chapter{Maquettage}
\section{Section}
\subsection{Sous-section}

\part{Conception UML}

\part{Implémentation}

\end{document}          
